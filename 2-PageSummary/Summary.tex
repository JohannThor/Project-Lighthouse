% v2-acmsmall-sample.tex, dated March 6 2012
% This is a sample file for ACM small trim journals
%
% Compilation using 'acmsmall.cls' - version 1.3 (March 2012), Aptara Inc.
% (c) 2010 Association for Computing Machinery (ACM)
%
% Questions/Suggestions/Feedback should be addressed to => "acmtexsupport@aptaracorp.com".
% Users can also go through the FAQs available on the journal's submission webpage.
%
% Steps to compile: latex, bibtex, latex latex
%
% For tracking purposes => this is v1.3 - March 2012

\documentclass[prodmode,acmtosem]{acmsmall} % Aptara syntax

% Package to generate and customize Algorithm as per ACM style
\usepackage[ruled]{algorithm2e}
\usepackage{textcomp}
\usepackage{gensymb}
\usepackage[utf8]{inputenc}
\usepackage[table]{xcolor}
\usepackage{natbib}
\usepackage{subfigure}
\usepackage{tabularx}

\renewcommand{\algorithmcfname}{ALGORITHM}
\SetAlFnt{\small}
\SetAlCapFnt{\small}
\SetAlCapNameFnt{\small}
\SetAlCapHSkip{0pt}
\IncMargin{-\parindent}

% Document starts
\begin{document}

% Page heads
\markboth{}{Dynamic obstacle mapping for the visually impaired using sensor fusion.}

% Title portion
\title{Dynamic obstacle mapping for the visually impaired using sensor fusion.}
\author{Johann Thor Kristthorsson
\affil{University College London}
Ifeanyi Ndu
\affil{University College London}
Veselin Pavlov
\affil{University College London}
Shuang Zhang
\affil{University College London}
}

\maketitle

\section{Project Context and Objectives}
Who is your client? What problem did you have to solve? Why is that problem interesting or important?

\section{Achievements}
What software and other artifacts have you delivered?
\section{Evaluation}
What evaluation have you done? What are the results of your evaluations?(Be specific; give numbers!)
\section{Impact}
What are the impacts of your work for your client and other stakeholders?
\section{Challenges}
Was your project subject to significant constraints and difficulties (technical, organisational, logistics, etc.)? How did you respond to these constraints and difficulties?
\section{Lessons Learnt}
What can other software engineers learn from your experience in this project? (take away messages)

\end{document}


