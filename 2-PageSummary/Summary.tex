% v2-acmsmall-sample.tex, dated March 6 2012
% This is a sample file for ACM small trim journals
%
% Compilation using 'acmsmall.cls' - version 1.3 (March 2012), Aptara Inc.
% (c) 2010 Association for Computing Machinery (ACM)
%
% Questions/Suggestions/Feedback should be addressed to => "acmtexsupport@aptaracorp.com".
% Users can also go through the FAQs available on the journal's submission webpage.
%
% Steps to compile: latex, bibtex, latex latex
%
% For tracking purposes => this is v1.3 - March 2012

\documentclass[prodmode,acmtosem]{acmsmall} % Aptara syntax

% Package to generate and customize Algorithm as per ACM style
\usepackage[ruled]{algorithm2e}
\usepackage{textcomp}
\usepackage{gensymb}
\usepackage[utf8]{inputenc}
\usepackage[table]{xcolor}
\usepackage{natbib}
\usepackage{subfigure}
\usepackage{tabularx}

\renewcommand{\algorithmcfname}{ALGORITHM}
\SetAlFnt{\small}
\SetAlCapFnt{\small}
\SetAlCapNameFnt{\small}
\SetAlCapHSkip{0pt}
\IncMargin{-\parindent}

% Document starts
\begin{document}

% Page heads
\markboth{}{Dynamic obstacle mapping for the visually impaired using sensor fusion.}

% Title portion
\title{Dynamic obstacle mapping for the visually impaired using sensor fusion.}
\author{Johann Thor Kristthorsson
\affil{University College London}
Ifeanyi Ndu
\affil{University College London}
Veselin Pavlov
\affil{University College London}
Shuang Zhang
\affil{University College London}
}
\maketitle

\section{Project Context and Objectives}
%Who is your client? What problem did you have to solve? Why is that problem interesting or important?
The Lighthouse team collaborated with Microsoft and the Guide Dog Association to produce applications which can improve the experience of visually impaired users. Since few environments and applications were designed for the huge amount of blind and partly sighted individuals it is difficult for them to move around especially in an unfamiliar environment without assistance. 


Microsoft is both the sponsor and the client in this project and particular field of study was to explore the use cases for wearable sensors in aiding the visually impaired in navigating an indoor environment. Also, the client wanted the team to use low-cost, off-the-shelf hardware in this project and to use a technique called sensor fusion to attain acceptable accuracy.

Finally, Lighthouse team created a platform for dynamic indoor obstacle mapping using wearable sensors. Deliverables included an Obstacle API which allowed for querying for obstacles in an area, an obstacle processing platform and an Android application that collects data and gives users feedback. After testing, the quality of the project met the client's expectations.

\section{Achievements}
%What software and other artifacts have you delivered?
\subsection{Obstacle mapping platform}
The team delivered a platform that gatheres data from sensors and pushes them to the cloud through a component called the Obstacle API which handles communication between data collection sensors and the processing platform. The obstacle API also provides an interface to query the platform for obstacles in a specified area. 

The data sent through to the Obstacle API is forwared to a processing pipeline that performs filtering, aggregation and other steps to produce estimated locations of obstacles in the area. The pipeline is implemented in Apache Kafaka and Apache Spark and uses sensor fusion techniques to increase the accuracy of the obstacle location estimation. 

An android application was created to query the Obstacle API to provide the user with details of obstacles in surrounding area. It shows the obstacles on the map that is centered on the location of the user.

\subsection{Secondary deliverables}
The MSc CS students in this team worked on collecting data from sensors and sending the readings through the Obstacle API. The also researched interaction and feedback mechanisms that the client wanted explored.

\section{Evaluation}
\subsection{Testing strategy}
Unit testing, load testing and integration testing were performed on each component in the project. Besides that, other testing, such as testing the response latency and the positioning accuracy were also used in this project to test the quality of the project.
\subsection{Results}
In the unit testing, the branch and statement coverage reaches 100\% according to EclEmma, on the components that were evaluated through unit tests. The result of load testing proved the platform met performance requirements. Upon performing the integration testing, results with sensor fusion showed desirable results compared to those without sensor fusion in both noisy and noiseless situations.
\section{Impact}
Microsoft
\begin{itemize}
\item[.] The Lighthouse Team has provided Microsoft with a platform to improve the experience of the visually impaired in indoor environments.
\item[.] Secondary deliverables provide valuable research to Microsoft that will help them make decisions in moving the Guide Dogs project further.
\end{itemize}
Visually impaired people
\begin{itemize}
\item[.] The Lighthouse project help them move around easily and avoid the obstacles with minimal assistance from others.
\item[.] It will also fortify them with confidence to explore the world and become more independent in their day to day lives.
\end{itemize}

\section{Challenges}
%Was your project subject to significant constraints and difficulties (technical, organisational, logistics, etc.)? How did you respond to these constraints and difficulties?
\begin{itemize}
\item[.] The initial plan of using hardware from Electrical Engineering students failed due to its poor quality. We moved the focus of the project away from that which proved unsuccessful.
\item[.] The testing was not performed with real data because the CS students didn't deliver expected data till the end of project. We tested the project with a simulator.
\item[.] Implementation of the simulator decoupled the work of the team from the CS students projects.
\item[.] The hardware requirements of the technologies used was underestimated initially. Quick upgrade of hardware mitigated the risk.
\item[.] Limitations in the way networking is done in Android required the team to rewrite Obstacle API.
\item[.] Sensor fusion techniques vary greatly in their use and implementation, the team analysed each and selected one that fits the purpose.
\end{itemize}


\section{Lessons Learnt}
\begin{itemize}
\item[.] Research into hardware requirements of frameworks should be done early.
\item[.] Identification of sensor fusion techniques is necessary and must fit the specific purpose of use and the type of inputs available.
\item[.] Managing risk is valuable in ensuring the project does not fail and decoupling of components provided risk mitigation.
\item[.] Investigate the limitations of platforms like Android early enough to mitigate the risk of wasted effort.
\item[.] New programming language may ease coding particularly Scala for Spark. Library support differs with regards to programming languages.
\end{itemize}

\end{document}


